% !TEX encoding = UTF-8
% !TEX TS-program = pdflatex
% !TEX root = ../Tesi.tex
% !TEX spellcheck = it-IT

%*******************************************************
% Introduzione
%*******************************************************
\cleardoublepage
\pdfbookmark{Introduzione}{introduzione}

\chapter*{Introduzione}
In questo documento present\'o alcuni algoritmi di pianificazione dinamica basati su Velocity Obstacle.
\begin{description}
\item[{\hyperref[cap:vo]{Il primo capitolo}}]
offre una visione d'insieme sul problema di Collision Avoidance e sul concetto di Collision Cone e Velocity Obstacle - VO.
\item[{\hyperref[cap:rvo]{Il secondo capitolo}}]
affronta il problema delle oscillazioni causate da Velocity Obstacle incorporato dalla natura reattiva degli altri robot - RVO.
\item[{\hyperref[cap:hrvo]{Il terzo capitolo}}]
affronta il problema delle oscillazioni causate da Reciprocal Velocity Obstacle dimezzando la responsabilit\'a della computazione della nuova velocit\'a - HRVO.
\item[{\hyperref[cap:orca]{Il quarto capitolo}}]
offre una spiegazione sul migliore algoritmo odierno di Collision Avoidance basato su Velocity Obstacle utilizzando le velocit\'a relative - ORCA.
\item[{\hyperref[cap:dvo]{Il quinto capitolo}}]
descrive l'implementazione della simulazione in Matlab utilizzando alcuni principi descritti precedentemente - DVO.
\end{description}


